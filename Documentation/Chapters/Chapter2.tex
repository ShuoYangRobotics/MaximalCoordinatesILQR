\section{Single rigid body with quaternion}

Consider a cubic base, noted as link 0, floating in space, we can define its state vector in the following form:
\begin{equation}
    \textbf{x} = \begin{bmatrix}
        \textbf{r} \\  \textbf{q} \\ \textbf{v} \\ \text{\boldmath$\omega$}
    \end{bmatrix} \in \mathbb{R}^{13}
\end{equation}
Where $\textbf{r}$ is the simplified form of $^{\mathcal{I}}\textbf{r}\in \mathbb{R}^3$ representing COM position in Inertial reference frame(world frame). $\textbf{q}$ is a unit quaternion representing rigid body's relative orientation to the world frame. $\textbf{v}$ and $\text{\boldmath$\omega$}$ are the linear velocity and angular velocity in body frame $\mathcal{L}_0$  respectively.

On the input side, we start by assuming full control over force $^{\mathcal{L}_0}\textbf{F} \in \mathbb{R}^3$ and torque $^{\mathcal{L}_0}\text{\boldmath$\tau$} \in \mathbb{R}^3$:
\begin{equation}
    \textbf{u} = \begin{bmatrix}
        \textbf{F} \\ \text{\boldmath$\tau$}
    \end{bmatrix} \in \mathbb{R}^{6}
\end{equation}
And then we adjust actuation by adjusting the $\textbf{B} \in \mathbb{R}^{6 \times 6}$ matrix.
\subsection{Dynamic modeling}
Linear velocity in world frame can be calculated by rotating body velocity vector $\textbf{v}$:
\begin{equation}
    \dot{\textbf{r}} = \textbf{q} \cdot \textbf{v}
\end{equation}

Quaternion rate $\dot{\textbf{q}}$:
\begin{equation}
    \dot{\textbf{q}} = \frac{1}{2}G(\textbf{q})\text{\boldmath$\omega$}
\end{equation}

Linear acceleration in body frame:
\begin{equation}
    \dot{\textbf{v}} = \frac{1}{m}\begin{bmatrix}
        \textbf{I}_3 & 0
    \end{bmatrix} \textbf{B}\textbf{u}
\end{equation}

Angular acceleration in body frame:
\begin{equation}
    \dot{\text{\boldmath$\omega$}} = \textbf{J}^{-1}\begin{bmatrix}
        0 & \textbf{I}_3
    \end{bmatrix} \textbf{B}\textbf{u}
\end{equation}

\subsubsection{Controlibility Analysis}\cite{jiang2020controllability}
This is trickier than I though with quaternions in the state, the regular rank method doesn't seems to work.

\subsection{Formulating a SQP}


\pagebreak